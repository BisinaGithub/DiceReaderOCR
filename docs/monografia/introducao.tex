\pagenumbering{arabic}
\setcounter{page}{8}   
\section{Introdução}

Os jogos de interpretação de papéis (Role-Playing Games - RPGs) têm conquistado cada vez mais espaço não só 
como forma de entretenimento, mas também como ferramenta pedagógica, terapêutica e social. Desde o lançamento
de \textit{Dungeons \& Dragons} em 1974, o RPG se consolidou como um gênero de jogos que une imaginação, 
cooperação e raciocínio lógico, promovendo experiências imersivas onde os jogadores atuam em conjunto na 
construção de narrativas.  \cite{peterson2012playing,hitchens2007roleplaying}

Uma das características marcantes desses jogos é o uso de dados físicos para determinar sucessos ou fracassos
de determinadas ações, como ataques, testes de habilidade ou eventos aleatórios, e, diferentemente dos jogos 
de tabuleiro convencionais, o RPG utiliza-se de uma diversidade de dados poliédricos 
— como d4, d6, d8, d10, d12 e d20 — que ampliam a variação probabilística das ações dos jogadores e tornam 
o jogo mais dinâmico e imprevisível. Além de sua usabilidade, esses dados se tornaram símbolos da cultura 
RPGista, muitas vezes personalizados com diferentes materiais, cores e gravuras, ou até eternizados em tatuagens, 
obras artísticas e coleções, reforçando o vínculo afetivo entre jogador e dado.\cite{hitchens2007roleplaying}

Apesar do crescimento de plataformas digitais que oferecem alternativas de rolagem automatizada, como
\textit{Roll20} ou \textit{Foundry VTT}, e até comandos de rolagem em bots de Discord, muitos jogadores ainda
preferem o uso dos dados físicos. Essa preferência se deve tanto ao valor simbólico e afetivo atribuído aos
objetos quanto à percepção de que a aleatoriedade física é mais justa ou autêntica do que a gerada 
computacionalmente. O som do dado rolando, o toque, o peso e até o suspense visual da face superior prestes a 
ser revelada são elementos sensoriais que contribuem para a ambientação do jogo e tornam a experiência 
mais sensorial, orgânica e memorável. \cite{zagal2017role}

No entanto, essa escolha estética e emocional traz um desafio prático. Em sessões que envolvem múltiplas
rolagens com dados, que muitas vezes podem conter múltiplos dados simultâneos — como testes de dano, ataques 
em área ou ações conjuntas — a contagem manual dos resultados obtidos pode se tornar demorada e propensa a erros. 
Esse tempo de cálculo afeta diretamente o ritmo narrativo da partida, especialmente em jogos com muitos 
participantes ou quando os personagens utilizados estejam com suas habilidades já em maior nível, aumentando
os números e resultando que dezenas de dados acabem sendo lançados ao mesmo tempo. Mesmo entre grupos experientes, 
a perda de fluidez entre ação e contagem numérica pode comprometer a imersão da experiência e gerar frustrações.

A crescente demanda por ferramentas que otimizem o tempo de jogo sem abrir mão da experiência de rolagem física
motivou este trabalho, que propõe uma solução baseada em visão computacional para automatizar a leitura visual dos
dados físicos durante uma rolagem, mantendo o aspecto tradicional do RPG com os dados físicos enquanto melhora 
a fluidez narrativa das sessões através da velocidade que um cálculo automático pode trazer. 
\cite{szeliski2010computer,ibm_cnn,datacamp_cnn}

A Figura~\ref{fig:dados-rpg} ilustra alguns exemplos de dados utilizados em sistemas de RPG, destacando a
variedade de formatos e faces numéricas. Essa diversidade, embora enriqueça o jogo, representa também uma
dificuldade computacional para quem deseja automatizar a leitura desses elementos visuais.

\begin{figure}[H]
    \centering
    \includegraphics[width=0.8\textwidth]{figuras/rpg-dice.JPG}
    \caption[Exemplos de dados poliédricos utilizados em jogos de RPG.]{
    \centering
    Exemplos de dados poliédricos utilizados em jogos de RPG.\par
    \textit{Foto por Danielly Vitória Rodrigues, 2025.}}
    \label{fig:dados-rpg}
\end{figure}

Partindo deste contexto, este trabalho tem como objetivo analisar a viabilidade do uso de Redes Neurais 
Convolucionais (CNNs) para o reconhecimento automático dos valores presentes nas faces superiores 
de dados físicos utilizados em jogos de RPG, a partir de um banco de imagens previamente separado e 
categorizado conforme o valor exibido na face superior dos dados. 
A proposta envolve o desenvolvimento e a avaliação experimental de um modelo de classificação multiclasse 
baseado em técnicas de visão computacional e aprendizado profundo, com foco na análise de seu desempenho 
em condições práticas simuladas. Para isso, será utilizada uma arquitetura de Rede Neural Convolucional (CNN) 
composta por múltiplas camadas convolucionais e de pooling, seguidas por camadas densas e uma camada de saída 
com função de ativação \textit{softmax}, adequada ao reconhecimento de múltiplas classes de saída — 
correspondentes aos valores possíveis nas faces dos dados de RPG. As CNNs foram escolhidas por demonstrarem
bastante eficiência em tarefas de reconhecimento de padrões visuais, especialmente em imagens onde existem 
cenários com variação de iluminação, ângulo, cor e textura. 
\cite{lecun1998gradient, goodfellow2016deep}

As Redes Neurais Convolucionais (CNNs) têm seu destaque em diversas áreas da inteligência artificial, como 
reconhecimento facial, diagnósticos por imagem, leitura de placas de veículos e OCR (Reconhecimento Óptico 
de Caracteres), sendo utilizadas amplamente devido ao seu sucesso em contextos que exigem robustez, 
acurácia e capacidade de generalização \cite{lecun1998gradient, goodfellow2016deep}. Por ter várias camadas 
organizadas em etapas, essa arquitetura consegue identificar detalhes visuais em diferentes níveis, o que a 
torna muito eficaz para reconhecer símbolos, números e formas em objetos tridimensionais. — como os dados 
físicos utilizados em jogos de RPG — mesmo sob condições variáveis de iluminação, ângulo ou resolução.

A principal motivação para o desenvolvimento deste estudo nasce da vivência pessoal do autor como mestre de RPG, 
especialmente ao observar, ao longo de diversas sessões, a forte preferência dos jogadores pelo uso de dados 
físicos. Mesmo com a ampla oferta de ferramentas digitais que simulam rolagens de forma prática e eficiente, 
é comum notar que muitos jogadores ainda optam pelos dados reais, seja pelo apego emocional, a sensação de adrenalina
e expectativa ao rolar um dado, ou pela conexão simbólica que esses objetos representam dentro da cultura do RPG.
No entanto, essa escolha, apesar de compreensível e até esperada, pode acabar trazendo desafios durante o jogo, 
principalmente em momentos em que múltiplos dados são rolados ao mesmo tempo. Nessas situações, a contagem 
manual dos resultados tende a consumir um tempo precioso, além de estar sujeita a erros que podem comprometer 
a precisão e, por consequência, a imersão e o ritmo narrativo da partida. Dessa forma, automatizar esse 
processo de leitura dos valores nos dados físicos surge como uma solução equilibrada: respeita e preserva a 
tradição dos jogos de mesa, mas aproveita os recursos da tecnologia para tornar a experiência mais fluida, 
dinâmica e eficiente para todos os participantes.

Essa proposta também se alinha com os avanços mais recentes no uso de sistemas de visão computacional, com 
destaque para o aumento da sua adesão em tecnologias assistivas. Um bom exemplo disso são os sistemas 
de reconhecimento óptico de caracteres (OCR), que hoje já são amplamente utilizados na conversão de documentos 
físicos em versões digitais editáveis, com uma precisão invejável. De forma semelhante, desenvolver um sistema 
capaz de identificar automaticamente os números presentes nas faces superiores dos dados físicos representa uma 
aplicação com potencial promissor. Tal recurso pode funcionar como uma importante ferramenta de apoio tanto para
mestres quanto para jogadores, especialmente para aquelas sessões realizadas de forma online ou híbrida, onde
a visualização e validação das rolagens físicas continua sendo valorizada como parte da experiência principal.

Com base nas necessidades identificadas ao longo da experiência prática com jogos de RPG de mesa, este trabalho 
tem como objetivo principal o desenvolvimento e a avaliação de um modelo computacional, fundamentado em redes 
neurais convolucionais (CNNs), capaz de reconhecer os valores exibidos nas faces superiores de dados físicos. 
Essa identificação será realizada a partir da análise de um conjunto de imagens previamente coletado e 
devidamente categorizado.

Para atingir esse objetivo, propõe-se então a execução das seguintes etapas:

\begin{itemize}
\item Construção de um banco de dados composto por imagens de diversos tipos de dados de RPG, capturadas sob 
diferentes ângulos de visão, exibindo valores diversos em sua face superior com o intuito de refletir a 
variabilidade presente em situações reais;
\item Projeção e treinamento de uma arquitetura de rede neural convolucional voltada especificamente para a 
tarefa de classificação multiclasse dos valores numéricos visíveis nas imagens;
\item  Avaliação do desempenho do modelo utilizando métricas quantitativas, como acurácia e taxa de erro, 
considerando diferentes cenários simulados;
\item Discussão voltada tanto à avaliação da aplicabilidade prática do sistema em sessões de RPG, quanto à 
identificação de pontos de melhoria e caminhos possíveis para o aprimoramento e a evolução do projeto em 
versões futuras.
\end{itemize}

A proposta citada parte, portanto, da oportunidade avaliada pela observação de uma limitação no uso dos dados 
físicos em sessões de RPG, propondo o uso de técnicas de aprendizado profundo como uma solução inovadora. 
Ao longo do trabalho, serão apresentados os fundamentos teóricos que embasam o projeto, a metodologia adotada para 
construção e validação do modelo, os resultados experimentais obtidos e, por fim, as considerações finais e 
sugestões para futuras melhorias.


\section{Fundamentação Teórica}

Este capítulo apresenta os fundamentos teóricos que sustentam o desenvolvimento do projeto, abrangendo os temas 
de jogos de RPG, visão computacional, aprendizado profundo, redes neurais convolucionais e reconhecimento de 
padrões visuais. O objetivo é fornecer ao leitor o embasamento necessário para compreender a relevância, os 
desafios e as soluções técnicas envolvidas no reconhecimento automático de valores em dados físicos de RPG.

\subsection{Jogos de Interpretação de Papéis (RPG)}

\subsubsection{História e Evolução dos RPGs}
Os jogos de interpretação de papéis, popularmente conhecidos como RPGs (do inglês, Role-Playing Games), constituem 
um dos gêneros lúdicos mais complexos e influentes da cultura popular contemporânea. Sua origem remonta a uma 
série de influências históricas, culturais e mecânicas que se entrelaçaram ao longo do século XX, até culminar
 na criação do primeiro sistema formal de RPG: Dungeons \& Dragons (D\&D), publicado em 1974 por Gary Gygax e 
 Dave Arneson.

Segundo Peterson (2012), os RPGs não surgiram isoladamente, mas como consequência direta da evolução dos jogos 
de guerra (wargames), que eram amplamente praticados por entusiastas militares e civis ao longo dos séculos 
XIX e XX. Esses jogos utilizavam miniaturas, mapas e regras complexas para simular combates históricos, sendo 
voltados principalmente para recreações táticas. A grande inovação de Arneson, e posteriormente de Gygax, foi 
substituir o controle de exércitos pelo controle de personagens individuais com papéis específicos e atributos 
únicos — introduzindo, assim, o elemento narrativo e subjetivo que diferencia os RPGs dos jogos de tabuleiro e 
estratégia.

Peterson também enfatiza o papel dos clubes de wargames norte-americanos no desenvolvimento das primeiras 
versões do que viria a se tornar o RPG moderno. Em particular, o cenário fictício de Blackmoor, criado por 
Dave Arneson, foi um dos primeiros mundos de campanha onde os jogadores podiam assumir papéis de personagens 
fixos e participar de aventuras continuadas.

A formalização dos RPGs como gênero distinto se deu com a publicação da primeira edição de Dungeons \& Dragons, 
que introduziu conceitos centrais como atributos numéricos (Força, Destreza, Inteligência, etc.), rolagem de 
dados poliédricos, níveis de experiência (XP), classes (como guerreiro, mago, ladrão) e sistemas de combate que 
integravam chance e estratégia. A proposta do jogo era que o Mestre (ou Dungeon Master, DM) assumisse o papel de 
narrador e árbitro, enquanto os jogadores assumiam personagens fictícios e colaboravam na construção de uma 
história interativa.

Hitchens e Drachen (2007) classificam os RPGs em quatro categorias principais: RPGs de mesa (como D\&D), RPGs ao 
vivo (Live Action Role-Playing Games, LARP), RPGs eletrônicos (Computer RPGs) e MUDs (Multi-User Dungeons). Essa 
taxonomia revela como o conceito de RPG se expandiu para diferentes mídias e contextos, mantendo como elemento 
unificador a ideia de interpretar um papel em um mundo imaginário sob certas regras.

A evolução dos RPGs também passou por uma transformação estética e temática significativa. Durante os anos 1980 e 
1990, sistemas alternativos ao D\&D surgiram, como GURPS (Generic Universal RolePlaying System), Vampire: The 
Masquerade (que inaugurou o gênero Storyteller), e Call of Cthulhu, baseado na obra de H. P. Lovecraft. Esses 
sistemas enfatizavam aspectos como interpretação dramática, dilemas morais e ambientações sombrias, ampliando o 
alcance narrativo dos jogos.

Nos anos 2000, os RPGs testemunharam uma popularização global com o advento de plataformas digitais, fóruns de 
discussão e, mais recentemente, com a ascensão de actual plays (sessões transmitidas ao vivo por streamers e 
criadores de conteúdo) e podcasts especializados. Títulos como Critical Role e canais como o Ordem Paranormal 
ajudaram a impulsionar uma nova geração de jogadores e expandiram o apelo do RPG como forma de entretenimento 
participativo.

A Figura~\ref{fig:rpg-evolucao} ilustra uma linha do tempo simplificada da evolução dos RPGs, desde suas origens 
nos wargames até sua presença contemporânea em plataformas digitais.

\begin{figure}[H]
\centering
\includegraphics[height=0.75\textheight, keepaspectratio]{figuras/Linha_do_Tempo_RPG_Mundo.png}
\caption[Linha do tempo simplificada e ilustrada da evolução dos RPGs]{
\centering
Linha do tempo simplificada e ilustrada da evolução dos RPGs, desde os wargames até os RPGs modernos.\par
\textit{Fonte: Silva, Gabriela Prates da; Prates da Silva, Eduardo (2020). História do RPG: linha do tempo. 
figshare. Figure. \url{https://doi.org/10.6084/m9.figshare.12493895.v5}.}}
\label{fig:rpg-evolucao}
\end{figure}

Em síntese, os RPGs representam uma forma única de jogo baseada na colaboração narrativa, na improvisação 
criativa e na imersão ficcional. Sua evolução demonstra não apenas uma transformação nas mecânicas de jogo, 
mas também um reflexo de mudanças culturais, sociais e tecnológicas. Ao longo das décadas, o RPG consolidou-se 
como ferramenta educacional, artística e terapêutica, além de manter seu lugar como um dos pilares do 
entretenimento alternativo.

\subsubsection{A Importância dos Dados Físicos no RPG}
Discussão sobre a função simbólica e prática dos dados físicos, incluindo o impacto da rolagem real na 
imersão e dinâmica do jogo.
\begin{itemize}
    \item Referência: \cite{hitchens2007roleplaying}
\end{itemize}

\subsection{Visão Computacional}

\subsubsection{Definição e Objetivos}

A Visão Computacional é um campo interdisciplinar da ciência e engenharia que busca desenvolver sistemas
capazes de extrair, interpretar e compreender informações a partir de imagens ou vídeos do mundo real. 
Inspirada nos mecanismos da visão humana, essa área utiliza algoritmos e técnicas matemáticas para permitir 
que computadores reconheçam padrões, identifiquem objetos e tomem decisões com base em dados visuais 
\cite{szeliski2010computer}.

Dentro do contexto da Engenharia da Computação, a Visão Computacional representa um elo entre o hardware e 
o software, exigindo conhecimento sólido em áreas como álgebra linear, processamento de sinais, inteligência 
artificial e arquitetura de computadores. A integração de câmeras com sistemas embarcados, o uso de GPUs 
para paralelismo em algoritmos de processamento de imagem e a aplicação de redes neurais profundas são 
exemplos de como esse campo dialoga com os fundamentos da engenharia \cite{bishop2006pattern}.

Os objetivos da Visão Computacional variam conforme o domínio de aplicação, mas incluem tarefas como:

\begin{itemize}
\item \textbf{Reconhecimento de padrões}: identificar rostos, placas de veículos, objetos ou caracteres;
\item \textbf{Segmentação de imagem}: separar regiões de interesse com base em características como cor, 
textura ou bordas;
\item \textbf{Rastreamento}: acompanhar o movimento de objetos ao longo do tempo em sequências de vídeo;
\item \textbf{Reconstrução tridimensional}: obter modelos 3D a partir de múltiplas imagens 2D;
\item \textbf{Análise de cenas}: compreender o contexto global de uma imagem, como a contagem de 
pessoas ou a detecção de anomalias.
\end{itemize}

O avanço das redes neurais convolucionais (CNNs), a disponibilidade de grandes conjuntos de dados rotulados 
e o aumento no poder computacional possibilitaram resultados significativos nos últimos anos, com aplicações 
em áreas como saúde (diagnóstico por imagem), indústria (inspeção automatizada), segurança (monitoramento 
por câmeras) e entretenimento (realidade aumentada).

Em projetos como este TCC, a Visão Computacional assume papel central, possibilitando a interface entre os 
dados físicos do mundo real (os dados de RPG) e o sistema computacional responsável por interpretá-los automaticamente.



\subsubsection{Etapas do Processamento de Imagens}

O processo de construção de um sistema de visão computacional envolve diversas etapas encadeadas, que vão desde 
a aquisição dos dados até a inferência dos resultados. Cada uma dessas etapas tem o seu papel no desempenho do modelo, 
exigindo que sejam feitas decisões técnicas pensadas com base nos objetivos do problema em questão.

A Figura~\ref{fig:cv-pipeline} ilustra o fluxo completo do trabalho em um sistema de visão computacional, trazendo 
destaque para os pontos de decisão e os vieses que podem influenciar negativamente o resultado final.

\begin{figure}[H]
\centering
\includegraphics[width=0.85\textwidth, height=0.35\textheight, keepaspectratio]{figuras/Computer-vision-pipeline.png}
\caption[Fluxo de um sistema de visão computacional]{
\centering
Fluxo de um sistema de visão computacional.\par
\textit{Fonte: Physics-Informed Computer Vision: A Review and Perspectives - Scientific Figure on ResearchGate.} \par
\raggedright
\textit{Available from: \href{https://www.researchgate.net/figure/Computer-vision-pipeline-showing-different-biases-and-different-points-of-physics_fig3_371136308}{https://www.researchgate.net/figure/Computer-vision-pipeline-showing-different-biases-and-different-points-of-physics\_fig3\_371136308} [accessed 5 Jun 2025]}
}
\label{fig:cv-pipeline}
\end{figure}

Esse fluxo é composto por cinco etapas, sendo elas:

\begin{itemize}
    \item \textbf{Aquisição de dados}: A etapa onde ocorre a coleta das imagens ou vídeos que serão utilizados no 
    treinamento do modelo. Os dados podem vir de fotos (2 dimensões), sensores com profundidade (3 dimensões) ou 
    streamings em tempo real. A qualidade e a diversidade dos dados capturados influenciam diretamente a capacidade 
    do modelo de fazer generalização.

    \item \textbf{Pré-processamento}: A etapa na qual são aplicadas transformações que preparam os dados para serem 
    utilizados pela rede neural. Dentre as operações mais comuns estão conversão para tons de cinza, normalização 
    dos valores do pixels, e técnicas de aumento de dados (\textit{data augmentation}). No contexto deste trabalho, 
    essa etapa incluiu o corte das bordas, redimensionamento padronizado das imagens e conversão de canais de cor RGB.

    \item \textbf{Projeto do modelo}: A etapa responsável pela arquitetura da rede convolucional que será utilizada, 
    como o número de camadas, quais filtros, suas funções de ativação e dimensões de entrada/saída. Modelos populares, 
    como GoogLeNet, ResNet e Mask-RCNN, oferecem diferentes estratégias para resolver tarefas como classificação, 
    detecção ou segmentação.

    \item \textbf{Treinamento do modelo}: A etapa que o modelo é exposto às imagens rotuladas e ajusta seus pesos 
    internos utilizando funções de perda como a entropia cruzada (\textit{cross-entropy}) ou erro quadrático médio. 
    O desempenho do modelo é medido com base em sua acurácia e na capacidade de generalizar para dados não vistos.

    \item \textbf{Inferência}: A etapa após o treinamento. Aqui o modelo é utilizado para fazer previsões sobre novos dados. 
    O tipo de inferência pode variar entre classificação, segmentação, detecção de objetos ou regressão, dependendo 
    do objetivo do projeto.

\end{itemize}

Além das etapas técnicas, a imagem também destaca três tipos de vieses que podem afetar os resultados:
\textbf{viés de observação} (dados enviesados), \textbf{viés de aprendizado} (modelo limitado em sua capacidade de 
generalização) e \textbf{viés indutivo} (pressupostos errados no design do modelo).

O entendimento de todas essas etapas é essencial para engenheiros da computação que desejam aplicar visão computacional 
a problemas do mundo real, como o reconhecimento automático de valores em dados físicos de RPG proposto neste trabalho.



\subsubsection{Reconhecimento Óptico de Caracteres (OCR)}
Breve explicação do OCR tradicional e comparação com abordagens baseadas em aprendizado profundo.
\begin{itemize}
    \item Referência: \cite{szeliski2010computer, bishop2006pattern}
\end{itemize}

\subsection{Aprendizado de Máquina e Aprendizado Profundo}

\subsubsection{Conceitos Básicos de Aprendizado de Máquina}
Aprendizado supervisionado, não supervisionado e por reforço; problemas de classificação.
\begin{itemize}
    \item Referência: \cite{bishop2006pattern}
\end{itemize}

\subsubsection{Avanços com Aprendizado Profundo (Deep Learning)}
O surgimento das redes neurais profundas e suas vantagens em problemas com grande volume de dados e 
complexidade visual.
\begin{itemize}
    \item Referência: \cite{goodfellow2016deep}
\end{itemize}

\subsection{Redes Neurais Convolucionais (CNNs)}

\subsubsection{Estrutura Geral de uma CNN}
Apresentação da arquitetura típica: camadas convolucionais, pooling, flatten e densas.
\begin{itemize}
    \item Referência: \cite{lecun1998gradient, goodfellow2016deep}
\end{itemize}

\subsubsection{Funcionamento das Camadas Convolucionais}
Filtro, stride, padding e extração de características espaciais.
\begin{itemize}
    \item Referência: \cite{lecun1998gradient, datacamp_cnn}
\end{itemize}

\subsubsection{Pooling e Redução Dimensional}
Funções de pooling (max, average) e sua função na redução de sobreajuste.
\begin{itemize}
    \item Referência: \cite{datacamp_cnn}
\end{itemize}

\subsubsection{Funções de Ativação e Camadas Densas}
Comparação entre ReLU, Sigmoid e Softmax. Papel das camadas densas e da saída.
\begin{itemize}
    \item Referência: \cite{goodfellow2016deep}
\end{itemize}

\subsubsection{Aplicações Práticas das CNNs}
Aplicações em reconhecimento facial, OCR, leitura de placas, diagnóstico por imagem.
\begin{itemize}
    \item Referência: \cite{goodfellow2016deep, lecun1998gradient}
\end{itemize}

\subsection{Aplicações Relacionadas: Reconhecimento de Dígitos e Objetos}

\subsubsection{Caso Clássico: LeNet-5 e MNIST}
Modelo LeNet-5 de LeCun e sua aplicação no MNIST para reconhecimento de dígitos manuscritos.
\begin{itemize}
    \item Referência: \cite{lecun1998gradient}
\end{itemize}

\subsubsection{Projetos Semelhantes}
Análise de projetos com reconhecimento de dados, OCR em contextos semelhantes, projetos com CNNs aplicadas 
a objetos tridimensionais e datasets customizados.
(Adaptar conforme trabalhos encontrados na revisão de literatura.)

\subsection{Considerações Finais}

Síntese dos principais conceitos abordados, reforçando o embasamento teórico que sustenta a metodologia 
aplicada no presente trabalho.
