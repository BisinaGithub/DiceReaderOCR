\section{Introdução}

Os jogos de interpretação de papéis (Role-Playing Games - RPGs) têm conquistado cada vez mais espaço não só 
como forma de entretenimento, mas também como ferramenta pedagógica, terapêutica e social. Desde o lançamento
de \textit{Dungeons \& Dragons} em 1974, o RPG se consolidou como um gênero de jogos que une imaginação, 
cooperação e raciocínio lógico, promovendo experiências imersivas onde os jogadores atuam em conjunto na 
construção de narrativas.  \cite{peterson2012playing,hitchens2007roleplaying}

Uma das características marcantes desses jogos é o uso de dados físicos para determinar sucessos ou fracassos
de determinadas ações, como ataques, testes de habilidade ou eventos aleatórios, e, diferentemente dos jogos 
de tabuleiro convencionais, o RPG utiliza-se de uma diversidade de dados poliédricos 
— como d4, d6, d8, d10, d12 e d20 — que ampliam a variação probabilística das ações dos jogadores e tornam 
o jogo mais dinâmico e imprevisível. Além de sua usabilidade, esses dados se tornaram símbolos da cultura 
RPGista, muitas vezes personalizados com diferentes materiais, cores e gravuras, ou até eternizados em tatuagens, 
obras artísticas e coleções, reforçando o vínculo afetivo entre jogador e dado.\cite{hitchens2007roleplaying}

Apesar do crescimento de plataformas digitais que oferecem alternativas de rolagem automatizada, como
\textit{Roll20} ou \textit{Foundry VTT}, e até comandos de rolagem em bots de Discord, muitos jogadores ainda
preferem o uso dos dados físicos. Essa preferência se deve tanto ao valor simbólico e afetivo atribuído aos
objetos quanto à percepção de que a aleatoriedade física é mais justa ou autêntica do que a gerada 
computacionalmente. O som do dado rolando, o toque, o peso e até o suspense visual da face superior prestes a 
ser revelada são elementos sensoriais que contribuem para a ambientação do jogo e tornam a experiência 
mais sensorial, orgânica e memorável. \cite{zagal2017role}

No entanto, essa escolha estética e emocional traz um desafio prático. Em sessões que envolvem múltiplas
rolagens com dados, que muitas vezes podem conter múltiplos dados simultâneos — como testes de dano, ataques 
em área ou ações conjuntas — a contagem manual dos resultados obtidos pode se tornar demorada e propensa a erros. 
Esse tempo de cálculo afeta diretamente o ritmo narrativo da partida, especialmente em jogos com muitos 
participantes ou quando os personagens utilizados estejam com suas habilidades já em maior nível, aumentando
os números e resultando que dezenas de dados acabem sendo lançados ao mesmo tempo. Mesmo entre grupos experientes, 
a perda de fluidez entre ação e contagem numérica pode comprometer a imersão da experiência e gerar frustrações.

A crescente demanda por ferramentas que otimizem o tempo de jogo sem abrir mão da experiência de rolagem física
motivou este trabalho, que propõe uma solução baseada em visão computacional para automatizar a leitura visual dos
dados físicos durante uma rolagem, mantendo o aspecto tradicional do RPG com os dados físicos enquanto melhora 
a fluidez narrativa das sessões através da velocidade que um cálculo automático pode trazer. 
\cite{szeliski2010computer,ibm_cnn,datacamp_cnn}

A Figura~\ref{fig:dados-rpg} ilustra alguns exemplos de dados utilizados em sistemas de RPG, destacando a
variedade de formatos e faces numéricas. Essa diversidade, embora enriqueça o jogo, representa também uma
dificuldade computacional para quem deseja automatizar a leitura desses elementos visuais.

\begin{figure}[H]
    \centering
    \includegraphics[width=0.8\textwidth]{figuras/rpg-dice.JPG}
    \caption[Exemplos de dados poliédricos utilizados em jogos de RPG.]{%
    \centering
    Exemplos de dados poliédricos utilizados em jogos de RPG.\par
    \textit{Foto por Danielly Vitória Rodrigues, 2025.}}
    \label{fig:dados-rpg}
\end{figure}

Partindo deste contexto, este trabalho tem como objetivo analisar a viabilidade do uso de Redes Neurais 
Convolucionais (CNNs) para o reconhecimento automático dos valores presentes nas faces superiores 
de dados físicos utilizados em jogos de RPG, a partir de um banco de imagens previamente separado e 
categorizado conforme o valor exibido na face superior dos dados. 
A proposta envolve o desenvolvimento e a avaliação experimental de um modelo de classificação multiclasse 
baseado em técnicas de visão computacional e aprendizado profundo, com foco na análise de seu desempenho 
em condições práticas simuladas. Para isso, será utilizada uma arquitetura de Rede Neural Convolucional (CNN) 
composta por múltiplas camadas convolucionais e de pooling, seguidas por camadas densas e uma camada de saída 
com função de ativação \textit{softmax}, adequada ao reconhecimento de múltiplas classes de saída — 
correspondentes aos valores possíveis nas faces dos dados de RPG. As CNNs foram escolhidas por demonstrarem
bastante eficiência em tarefas de reconhecimento de padrões visuais, especialmente em imagens, especialmente 
onde existem cenários com variação de iluminação, ângulo, cor e textura. 
\cite{lecun1998gradient, goodfellow2016deep}

As Redes Neurais Convolucionais (CNNs) têm seu destaque em diversas áreas da inteligência artificial, como 
reconhecimento facial, diagnósticos por imagem, leitura de placas de veículos e OCR (Reconhecimento Óptico 
de Caracteres), sendo utilizadas amplamente devido ao seu sucesso em contextos que exigem robustez, 
acurácia e capacidade de generalização \cite{lecun1998gradient, goodfellow2016deep}. Por ter várias camadas 
organizadas em etapas, essa arquitetura consegue identificar detalhes visuais em diferentes níveis, o que a 
torna muito eficaz para reconhecer símbolos, números e formas em objetos tridimensionais. — como os dados 
físicos utilizados em jogos de RPG — mesmo sob condições variáveis de iluminação, ângulo ou resolução.

A principal motivação para o desenvolvimento deste estudo nasce da vivência pessoal do autor como mestre de RPG, 
especialmente ao observar, ao longo de diversas sessões, a forte preferência dos jogadores pelo uso de dados 
físicos. Mesmo com a ampla oferta de ferramentas digitais que simulam rolagens de forma prática e eficiente, 
é comum notar que muitos jogadores ainda optam pelos dados reais, seja pelo apego emocional, a sensação de adrenalina
e expectativa ao rolar um dado, ou pela conexão simbólica que esses objetos representam dentro da cultura do RPG.
No entanto, essa escolha, apesar de compreensível e até esperada, pode acabar trazendo desafios durante o jogo, 
principalmente em momentos em que múltiplos dados são rolados ao mesmo tempo. Nessas situações, a contagem 
manual dos resultados tende a consumir um tempo precioso, além de estar sujeita a erros que podem comprometer 
a precisão e, por consequência, a imersão e o ritmo narrativo da partida. Dessa forma, automatizar esse 
processo de leitura dos valores nos dados físicos surge como uma solução equilibrada: respeita e preserva a 
tradição dos jogos de mesa, mas aproveita os recursos da tecnologia para tornar a experiência mais fluida, 
dinâmica e eficiente para todos os participantes.

Essa proposta também se alinha com os avanços mais recentes no uso de sistemas de visão computacional, com 
destaque para o aumento da sua adesão em tecnologias assistivas. Um bom exemplo disso são os sistemas 
de reconhecimento óptico de caracteres (OCR), que hoje já são amplamente utilizados na conversão de documentos 
físicos em versões digitais editáveis, com uma precisão invejável. De forma semelhante, desenvolver um sistema 
capaz de identificar automaticamente os números presentes nas faces superiores dos dados físicos representa uma 
aplicação com potencial promissor. Tal recurso pode funcionar como uma importante ferramenta de apoio tanto para
mestres quanto para jogadores, especialmente para aquelas sessões realizadas de forma online ou híbrida, onde
a visualização e validação das rolagens físicas continua sendo valorizada como parte da experiência principal.

Com base nas necessidades identificadas ao longo da experiência prática com jogos de RPG de mesa, este trabalho 
tem como objetivo principal o desenvolvimento e a avaliação de um modelo computacional, fundamentado em redes 
neurais convolucionais (CNNs), capaz de reconhecer os valores exibidos nas faces superiores de dados físicos. 
Essa identificação será realizada a partir da análise de um conjunto de imagens previamente coletado e 
devidamente categorizado.

Para atingir esse objetivo, propõe-se então a execução das seguintes etapas:

\begin{itemize}
\item Construção de um banco de dados composto por imagens de diversos tipos de dados de RPG, capturadas sob 
diferentes ângulos de visão, exibindo valores diversos em sua face superior com o intuito de refletir a 
variabilidade presente em situações reais;
\item Projeção e treinamento de uma arquitetura de rede neural convolucional voltada especificamente para a 
tarefa de classificação multiclasse dos valores numéricos visíveis nas imagens;
\item  Avaliação do desempenho do modelo utilizando métricas quantitativas, como acurácia e taxa de erro, 
considerando diferentes cenários simulados;
\item Discussão voltada tanto à avaliação da aplicabilidade prática do sistema em sessões de RPG, quanto à 
identificação de pontos de melhoria e caminhos possíveis para o aprimoramento e a evolução do projeto em 
versões futuras.
\end{itemize}

A proposta citada parte, portanto, da oportunidade avaliada pela observaçã de uma limitação no uso dos dados físicos 
em sessões de RPG, propondo o uso de técnicas de aprendizado profundo como uma solução inovadora. 
Ao longo do trabalho, serão apresentados os fundamentos teóricos que embasam o projeto, a metodologia adotada para 
construção e validação do modelo, os resultados experimentais obtidos e, por fim, as considerações finais e 
sugestões para futuras melhorias.


\vspace{1cm}
DAQUI PRA BAIXO NÃO TEM NADA NOVO


\section{Fundamentação Teórica}

Este capítulo apresenta os principais conceitos e fundamentos
que sustentam o desenvolvimento deste trabalho. Serão abordados
os aspectos históricos e técnicos do RPG, os tipos de dados
utilizados nesses jogos, os fundamentos de visão computacional,
o reconhecimento de padrões, as redes neurais artificiais e,
em especial, as redes neurais convolucionais (CNNs), tecnologia
central desta pesquisa.

\subsection{Role-Playing Games (RPGs)}

Os jogos de interpretação de papéis, conhecidos como Role-Playing
Games (RPGs), são uma forma de jogo narrativo em que os
participantes assumem papéis fictícios e interagem em um mundo
imaginário, muitas vezes guiados por um narrador ou mestre
\cite{hitchens2007roleplaying}. Esses jogos ganharam notoriedade
não apenas como forma de lazer, mas também como ferramentas de
ensino, inclusão social e desenvolvimento emocional, ao promoverem
criatividade, resolução de problemas, empatia e trabalho em equipe.

Além disso, os RPGs físicos — aqueles que envolvem livros, fichas
e dados — permanecem como uma das formas mais tradicionais e
culturalmente ricas de interação nesse gênero, mesmo em um cenário
atual fortemente digitalizado. A experiência tátil, a interpretação
ao vivo e a imprevisibilidade dos dados físicos contribuem para
uma experiência singular e imersiva, valorizada por seus praticantes.

\subsection{Dados em Jogos de RPG}

Diferentemente dos jogos de tabuleiro convencionais, os RPGs
utilizam uma variedade de dados poliédricos para simular testes
de atributos e ações. Entre os tipos mais comuns estão o d4
(tetraedro), d6 (cubo), d8 (octaedro), d10 (decágono), d12
(dodecágono) e d20 (icosaedro), sendo cada um empregado conforme
a mecânica do sistema de jogo.

A leitura dos valores apresentados nas faces superiores desses
dados é fundamental para o progresso da narrativa. No entanto,
essa tarefa torna-se complexa em sessões com múltiplas rolagens
simultâneas. Em especial, sistemas que utilizam muitos dados para
calcular danos, acertos ou testes em grupo enfrentam perda de tempo
significativa, além de suscetibilidade a erros na leitura manual.

Essa dificuldade cresce conforme a diversidade de dados utilizados.
Dados com faces pequenas ou com símbolos estilizados, como os d4
e alguns d10, tornam a interpretação visual ainda mais desafiadora,
o que reforça a necessidade de soluções que automatizem essa
leitura sem comprometer a fidelidade da experiência física.

\subsection{Visão Computacional}

A visão computacional é uma área da inteligência artificial que
busca desenvolver métodos para que sistemas computacionais possam
interpretar e compreender informações visuais do mundo real
\cite{szeliski2010computer}. Isso envolve desde a captação de imagens,
passando por seu processamento e análise, até a extração de dados
relevantes para tomada de decisão automatizada.

No presente trabalho, a visão computacional é empregada com o
intuito de detectar, identificar e interpretar os valores apresentados
na face superior dos dados físicos de RPG. Isso requer o domínio
de técnicas como pré-processamento de imagens, filtragem de ruído,
segmentação, detecção de contornos e extração de características
visuais que serão posteriormente classificadas por modelos baseados
em aprendizado profundo.

\subsection{Reconhecimento de Padrões}

O reconhecimento de padrões é o campo responsável por identificar
regularidades e estruturas em dados brutos, com o objetivo de
classificá-los ou agrupá-los de forma significativa
\cite{bishop2006pattern}. Ele é amplamente aplicado em tarefas como
reconhecimento facial, leitura de textos (OCR), análise de expressões
faciais e identificação de objetos em imagens.

No contexto deste trabalho, o reconhecimento de padrões é a base
para que se possa distinguir, de forma automática, os valores
apresentados nas faces dos dados físicos. Essa tarefa exige que
o sistema seja capaz de lidar com variações de iluminação, ângulo,
foco e tipo de fonte ou símbolo presente nos dados, o que torna
indispensável o uso de modelos robustos e adaptáveis, como as
redes neurais convolucionais.

\subsection{Aprendizado de Máquina e Redes Neurais Artificiais}

O aprendizado de máquina (machine learning) é uma subárea da
inteligência artificial focada no desenvolvimento de algoritmos
capazes de aprender padrões e realizar previsões ou classificações
com base em dados. Entre os modelos mais poderosos e amplamente
utilizados nessa área estão as redes neurais artificiais
\cite{goodfellow2016deep}.

Essas redes são compostas por camadas de unidades chamadas neurônios
artificiais, que processam informações em conjunto. Por meio do
treinamento supervisionado, elas ajustam seus parâmetros internos
com base em conjuntos de dados rotulados, adquirindo a capacidade
de realizar inferências em dados novos e não vistos.

Tais características fazem das redes neurais ferramentas valiosas
em aplicações complexas, como reconhecimento de fala, tradução
automática, diagnósticos médicos e, principalmente, análise de imagens.

\subsection{Redes Neurais Convolucionais (CNNs)}

As Redes Neurais Convolucionais (Convolutional Neural Networks —
CNNs) são uma arquitetura especializada de redes neurais artificiais,
voltada especificamente ao processamento de imagens e dados com
estrutura espacial \cite{lecun1998gradient}. Sua principal
característica é a capacidade de extrair automaticamente padrões
relevantes em imagens, sem necessidade de pré-processamento manual.

As CNNs operam por meio de camadas convolucionais que aplicam
filtros (ou kernels) sobre a imagem de entrada, capturando
características locais como bordas, formas e texturas. Essas
informações são progressivamente refinadas ao longo das camadas,
permitindo que o modelo aprenda representações hierárquicas de
alta complexidade.

Além das camadas convolucionais, as CNNs costumam incluir:
\begin{itemize}
\item \textbf{Camadas de Pooling}: responsáveis por reduzir a
dimensionalidade das representações intermediárias,
agregando informações locais e promovendo invariância a
pequenas mudanças na entrada;
\item \textbf{Camadas Densas}: totalmente conectadas, utilizadas
para combinar as características extraídas e realizar a
classificação final;
\item \textbf{Camada de Saída}: fornece a predição do modelo, que
pode assumir diferentes formatos conforme a tarefa
(classificação binária, multiclasse, etc.).
\end{itemize}

O uso de CNNs neste trabalho permite que o sistema reconheça os
dígitos nas faces superiores dos dados de RPG com alto grau de
robustez, mesmo em condições adversas, como iluminação irregular,
oclusão parcial ou variações na distância da câmera.

\subsection{Aplicações Relevantes de CNNs}

As redes neurais convolucionais têm sido empregadas com sucesso
em uma ampla gama de aplicações modernas. Dentre as mais relevantes,
destacam-se:
\begin{itemize}
\item \textbf{Reconhecimento Facial}: utilizado em autenticação
biométrica e sistemas de vigilância;
\item \textbf{Diagnóstico Médico por Imagem}: como na detecção de
tumores em exames de imagem, por exemplo, em mamografias
ou tomografias;
\item \textbf{OCR (Reconhecimento Óptico de Caracteres)}:
leitura automática de documentos e placas de veículos;
\item \textbf{Sistemas de Navegação e Robótica}:
reconhecimento de sinais visuais em ambientes dinâmicos;
\item \textbf{Indústria e Logística}: inspeção automatizada de
produtos e leitura de códigos visuais.
\end{itemize}

Tais aplicações comprovam a versatilidade e eficácia das CNNs,
e embasam sua escolha como tecnologia central para a proposta deste
TCC, que busca integrar o potencial do aprendizado profundo ao
contexto lúdico dos jogos de RPG.
